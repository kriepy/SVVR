\documentclass[12pt]{article}

\author{Kristin Rieping (10252428) \and Toby \and Mihai \and Robrecht}
\date{today}

\begin{document}
\maketitle

\begin{abstract}
In this report we desribe the visualisation of the calcification in the coral balanophyllia europaea

The data is in form of dicom and is gained with a CTscan. We build a pipeline in where the data of a batch corals is seperated, visualised and analysed.
\end{abstract}

\section{Introduction}
what is given:\\
2 batches a 10 corals that are scanned in a CT scanner, one batch normal, other in acified water\\
data gives density value for each voxel\\
with this data we want to visualise the calcification in the corals\\


\subsection{Coral}
-some information about the coral:\\
2 batches, acified water and not
location??
bleached corals, so they are dead

\subsection{CTscan information}

dicom files\\
contains density values for every voxel\\


\section{The Rendering Pipeline}
first seperating the data, so that every coral can be shown alone\\
-which 

then measuring the volume in 1 way, choose which\\
getting the density 

also give vtk pipeline

\section{Evaluation of methods}
volume estimation with cubes, surfaces and integrating the density histogram. They give similar results but are different to the measured volume. counting the voxels is computational to expensive with VTK.
compare with measured volumes:
they differ because there is some massa which is not seen in the CT scan OR there are empty spaces in the coral which is also taken into account while measuring the volume

\section{Results}
values from -1000 to 2100(later)
show visualisations
show density graphs
compare with measured(real) values
\section{Conclusion}


\end{document}